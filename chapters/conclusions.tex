The aim of this portfolio was to prove professionalism within education at \acrshort{fhict} through didactics, testing and \acrshort{tel}, as well as research. 
As a pedagogue, didactical skills are continuously developed to adapt to the scholars and institute demands. 
Furthermore, a pedagogue should adapt education according to the industry's (e.i. enterprises or organizations) demands to train industry prepared professionals.
\\\\
The evidence presented in this portfolio is the result of the past three years of activity within the institute. Having performed several roles already (e.i. lecturer, assessor and internship tutor), the task of didactics development and research presented in this work were well received. 
As part of the didactics, testing and \acrshort{tel}, the redevelopment of course materials for a fourth year ICT \& Software Engineering elective was presented. 
This redevelopment motivation was well backed up by personal experience as well as student feedback. 
The course was fully revised using the \acrshort{soo} model, including revision of learning outcomes according to the Learning Outcomes Manual~\cite{learningoutcomesmanual}.
Furthermore, the \acrshort{fhict} assessment policy~\cite{FHICTAssesmentPolicy} was followed in the creation of the course assessment. 
The course was then physically published through \acrshort{tel} means (e.i. lecture slides, course notes and exercise files) onto \gls{canvas}, \acrshort{fhict}'s online \acrshort{lms}.
Parallelly, the research task was undertaken where research expertise as well as research supervision activity in internal and external projects was presented. The task of supervision was described through examples, as well as assessments. Furthermore, as research expertise, a study of student performance based on developed course material was presented. 
\\\\
Overall, I believe this work has improved my didactical skills and has made me a better \acrshort{fhict} pedagogue. First of all, learning about various sources of information that the institute offers about the education has been eye opening. While reading about policies and rules has helped me become more confident in my employed didactics and the quality of education, it has also paved a strong background about curriculum development and course design tasks which I am willing to shoulder. Furthermore, research will always be my passion, thus, I would continuously look for opportunities to improve personally and also improve education.