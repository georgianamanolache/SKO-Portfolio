Normally, lessons are given face-to-face, where the teacher conducts instruction with the aid of digital tools. 
Furthermore, information is also made available digitally through online platforms.   
On overview of these tools follows.

\subsection{Canvas}

At \acrshort{fhict}, \Gls{canvas} is the the primary \acrfull{lms}, gradually replacing the former platform \Gls{sharepoint}. 
\acrshort{lms}s are defined as a type of software application where programmed instructions drive all learning activities; furthermore, they act as a repository where learning resources can be stored~\cite{lms2020}. 
In particular, \Gls{canvas} offers a series of features, from information communication (e.i. theory and assignment publication) to assessment and progress tracking by means of grade book (e.i. formative and summative indications). 
Moreover, \Gls{canvas} is highly customisable, making it a flexible environment for both the teachers and students.
I regularly utilize these feature in my teaching activity and I also guide students in navigating the platform. 
Having experience with the former \acrshort{lms}, an advantage of \Gls{canvas} is that it helps students to organize in a timely manner, providing notifications about upcoming assignments and deadlines for enrolled courses. 
This feature does not exist in \Gls{sharepoint}, teachers usually having to revert to publishing announcements onto the main page. 
However, this was not efficient as students would see announcements of courses they were not enrolled to and would be overwhelmed with redundant information.
Another advantage of \Gls{canvas} is that it provides a centralized assessment system. 
Previously, assignments were handed through emails and the assessment was published as a \Gls{mexcel} spreadsheet. 
Furthermore, due to the disrupted communication, feedback was less digital. 
An example of an assignment assessment and feedback with \Gls{canvas} is shown in~\cref{canvas_assessment_feedback}.

\subsection{Other tools}
Instructions are primarily guided through \Gls{mppt} slides. 
I often invest time animating components in my presentations, especially to emphasize a process or a method. 
As \citet{ppt2011}  study showed, lower textual density in slides and added non-textual elements appear to stimulate positive student feedback. 
Furthermore, I also utilize slides to engage students and initiate discussion. 
Some example slides are shown in~\cref{appendices:slide_assignment_info}.\\\\
To stimulate students further, I frequently run theory quizzes using \Gls{kahoot}. This is a web-based student response system that engages students through game-like premade or impromptu quizzes, discussions and surveys and can be easily accessed from any device such as iPad, Android device, or Chromebook~\cite{kahoot2015}. According~\citet{icard2014} and many other studies, game-based learning is considered a best practice in education. An example of a \Gls{kahoot} quiz is shown in~\cref{appendices:kahoot}.\\\\
Assignments or document templates are usually written as a \Gls{mword} document and published as a \acrshort{pdf}. 
I also endorse \gls{latex}, another software system for text document editing, especially for more advanced students (e.i. internship and undergraduate students in the final bachelor's year). \gls{latex} is based on the philosophy that authors should be able to focus on the content of what they are writing without being distracted by its visual presentation~\cite{latex2014}, which is usually the case with other text editing tools like \Gls{mword}. 
To ease student's introduction to \gls{latex}, I created a document template which is publicly available in \Gls{overleaf}, as shown in~\cref{appendices:latex}.\\\\
For software development, the version control system \Gls{git} is usually used to track code changes. 
Students usually set up and administer a \Gls{git} repository and then a link is further shared. 
This tool is especially convenient for projects, either grouped or individual. For instance, while not mandatory, I endorsed \gls{git} to \acrshort{ale} students as a simpler way of sharing their source code for feedback, instead of uploading archived folders to \gls{canvas}.\\\\
Although Canvas offers communication channels, audio-visual communication is lacking; therefore, I decided to introduce another platform, \Gls{slack}. 
This idea evolved as a way to unite and stimulate student discussions on course topics in an online environment. 
Before the COVID-19 outbreak and the migration to online learning, I have used \Gls{slack} with all the four scholarly year students. First, I saw great opportunities to group students in one platform (other than WhatsApp and Facebook) and discuss solutions to \acrshort{ale} assignments. Then, I introduced \gls{slack} to my mentor first semester students as a platform to acquaint and discuss education related topics, also including, but not limited to, personal issues with respect to moving and living in the Netherlands. Currently this platform is replaced by \acrshort{fhict} with another officially administered platform for online lessons during the COVID-19, presented in the following section.  
