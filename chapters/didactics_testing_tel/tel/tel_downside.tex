Despite the rapid technological advancements, education is facing some problems.
First and foremost the internet as a source of information poses a threat to the learning behavior of students.
While the source of information is endless, it is not necessarily accredited.
Excessive learning individualization through online means may lead to the denial of teacher-student or professional-apprentice dialogue. 
For example, during internships many students get consumed into the online research and individualize their process, thus, missing consulting the company. This is especially the case for students performing internships during the COVID-19 pandemic, where regular activities at the company are carried out online.\\\\
Furthermore, new technology creates new issues in terms of cheating and plagiarism.
Although the internet has helped to accelerate the research process for many academics, it has also created easy access to methods of cheating and plagiarism for students~\cite{downside2006}.
For instance, students would rather copy already existing solutions from various online forums such as \gls{stackOverflow}, rather than spending time solving the problem themselves, thus, missing the practice. This is the case for many student in the \acrshort{ale} course for which there are solutions available online. Thought my experience in teaching this course, I identified many cases of fraud with an internal plagiarism check tool.\\\\
Apart from damaging practical skills such as paper writing or arithmetic, excessive use of technology also leads to reality disconnection. Students are dependent of their electronic devices; therefore, their attention to the surroundings decreases. While the lessons at~\acrshort{fhict} usually require the use of technology, it is students responsibility to avoid distractions and cooperate with the teacher for a good learning atmosphere. 

