As I consider myself a fast learner, I employ \acrshort{tel} effortlessly. Especially my recent experience as an alumna in similar education has fortified my \acrshort{tel} skills. 
Having used both \gls{canvas} and \gls{sharepoint} during my studies, identifying and implementing familiar features was straightforward. 
Even the generation of production courses within \gls{canvas} (e.i. generate a start-up version of a course for new students every semester, including new deadlines and updated course materials), although a new feature, was manageable by myself.
\\\\
\acrshort{tel} has definitely revolutionized education, allowing online means to learning even during a global crisis such as COVID-19. 
Although nowadays everyone is familiar with the online environment, to me this change came naturally as I have used online tools for education before. However, I believe education does not change because the world changes. 
If employed properly, education through online means may enhance the learning experience in ways traditional face-to-face education lacks~\cite{blended2020}. 
Throughout the COVID-19 crisis I adapted almost weekly to the student's feedback. 
One advantage I discovered was the ease of student engagement in problem solving questions. 
While before it was more difficult to share one student's problem and discuss it collectively, with the online means and the ability to share screen, students became proactive in asking advise questions and sharing ideas and solution. 
Another obvious advantage students also pointed out is the possibility of re-watching video recorded lectures for strengthening knowledge, a possibility that \acrshort{fhict} would like to keep even after the COVID-19 crisis.
\\\\
In a global pandemic technology has proven its advantages. However, there are always disadvantages to advancements in technology in general.
Even during online lessons, students may be easily distracted on their other screens and machines (e.i. phone, tablets). 
This is a common issue in the face-to-face lessons and many lecturers deal with it differently. 
For the online lessons, I always make sure to ask questions regularly to confirm student's active presence in the lessons. 
For the face-to-face lessons, I let the students know when the lessons are not mandatory and if they prefer to work on other tasks they may leave the classroom. 
Also, I pay very well attention to students and often walk through the class while lecturing.
Other issues such as plagiarism and accessing unaccredited information from the online sources I try to prevent and/or uncover the with internal plagiarism check tool. However, for the most part preventing is what I strive for and helping students to change their mind set.
