Although the course started with ten students, in the end only one student completed the course. After discussion, I concluded many students decided to join less time consuming courses and spend the rest of the time working on the side. While this was a situation that could not be predicted, the course being an elective after all, a comparison between the old and the new course materials cannot be statistically analysed anymore. However, a statistical analysis is shown in  \cref{chapter:research_expertise} containing an extensive study over the student performance on the basis of course material from 2017-2021.\\\\
Only the student that completed the assignment provided an evaluation over the new course material in written form. Based on the studnet's written evaluation, also shown in \cref{appendices:evaluation}, I concluded that the way the course manual (syllabus) was presented in \acrshort{ale}2 is preferred. Furthermore, as the student enjoyed the subject (joined both \acrshort{ale}1 and redesigned \acrshort{ale}2), I value this opinion highly. A remark the student highlighted is the need of more examples. These were already added in the new syllabus, but more detailed examples are expected. While I do not entirely agree with this recommendation, I understand the inconvenience the students go through in designing their own input examples. In fact, part of this project is that students create their own examples and test with each other the outputs. Given the fact that the \acrshort{ale}2 Fall 2020 student was left alone in the class, creating own examples and testing is less reassuring. However, I will consider this once again, and try to elaborate more examples for the next iteration.

