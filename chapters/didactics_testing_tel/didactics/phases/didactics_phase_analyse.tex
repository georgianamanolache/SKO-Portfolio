Initially, the needs of students and the professional field are considered. My personal experience with \acrfull{pie} through student internships is in fact one of the reasons to develop a writing skills module. Generally, \acrshort{pie}'s feedback on written tasks boils down to \textit{informality in style }and difficulty to provide a \textit{smooth flow of ideas}. Furthermore, many alumni indicate struggles in the writing tasks, especially during the graduation. I am myself a \acrshort{fhict} alumna and I recognize similar lack of information.
\\\\
In my opinion, the most suitable audience for the writing skill module workshop are Semester 3 and Semester 6 students, respectively.
Both semester are in fact before the two internships, which are the two study unites heavily dependent on writing tasks. Furthermore, the students already acquire prior knowledge about the required content of the writing tasks within the project of ICT \& Software Engineering Semester 2, tasks which are repeated in the following semester, during a similar project involving \acrshort{pie}. Furthermore, additionally to the practice and experience of prior semesters, with Semester 6 the focus on research would increase, thus, so would the complexity of the the writing tasks.
\\\\
The writing module can be integrated in ICT \& Software Engineering Semester 3 as supportive workshop. While the development of Semester 6 is not yet started, I predict a similar context.
%Initially, a general picture of the course current situation was depicted, analyzing the \textit{target audience}, their \textit{needs} and the \textit{context} in which education is being developed. 
%Some example personas described in~\cref{appendices:personas} were composed to better understand the target audience.
%While the course is a fourth year elective, throughout my experience as an \acrshort{ale} teacher students that did not manage to find an internship project (thus, are half way in their third year) are usually joining electives to not waste an academic semester. 
%However, as a forth year ICT \& Software Engineering elective, \acrshort{ale} targets young professionals with a strong technical background, stimulated by applied mathematics and software engineering. 
%Preferably, \acrshort{ale} students should have studied in the ICT \& Software Engineering or ICT \& Technology profile and should have completed at least the first two academic years. For more details about the \acrshort{fhict} program see \cref{appendices:curriculum_structure}. 
%Furthermore, the needs of both the students and the professional field were considered, where solid software engineering competences were recognized essential, especially as an \acrshort{ict} graduate.
%Therefore, course material development would not imply changing the course structure, but \textit{clarify and enrich the learning experience}.
%As a consequence to the analysis, the context (such as instructional materials, learning environment) was realized and it is further described in the next section.