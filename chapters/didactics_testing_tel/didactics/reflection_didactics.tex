Developing course material has been an enlightening experience. 
While I vaguely knew about the faculty's policy and employed didactical methods, I learned about many new aspects that define education within \acrshort{fhict}. 
\\\\
Firstly, learning about the HBO-i framework has helped me better understand the expected competencies at a certain time in the study.
While I was aware of the increasing level of proficiency in each semester having attended similar education, previously, I only assumed the expected competencies and skills. The HBO-i framework offers an easy to follow separation of professional tasks and expected skills, which can be easily translated into the a course.  
Furthermore, the framework has helped me improve the quality of the already existing learning outcomes, as well as introducing a new learning outcome, namely documentation, essential in the final year of \acrshort{hbo} education. 
\\\\
Secondly, learning about a \acrshort{fhict} standardized procedure of defining learning outcomes has helped me provide consistent course design. 
I believe that consistency is key to delivering education. 
Students acquaint with certain standards which, if consistent, can ease learning within an institution's education, as students would expect similar requirement formulations and definitions, although on different topics and difficulty levels. 
While this is not the case in the old curriculum, I believe that the new curriculum is respecting these learning outcome standards and provides consistent education.
\\\\
Another important insight I learned while developing course material is the \acrshort{4cid} model for conveying knowledge through theory and practical materials. 
As I quickly realized I was already adopting to this model in my way of teaching, I was pleased to learn it is applied quite often within \acrshort{fhict}, especially in the new curriculum.
While the learning model may vary as the students progress in their study, I believe at least in the first academic year is ideal to have consistent education.
In addition, while researching didactics within \acrshort{fhict}, I also learn about the demand-based variant. 
This gave me some basic knowledge of the structure of this variant, which helps me better explain first year course-based English stream students what is the difference between the two variants.
\\\\
Lastly, I learned about the \acrshort{soo} model and how to properly conduct course development. 
I have re-developed many courses (e.i. theory slides, demo examples, mock exams and exams content) throughout the four \acrshort{fhict} academic years, always thriving to provide learning materials according to the students feedback.
However, I think having such a model, once again, provides consistency in the institute's education and eases a student's experience throughout the learning process. 
\\\\
Overall, I enjoyed and the entire course development process and I have already been offered to contribute to the development of the new curriculum semester six, which I accepted enthusiastically.