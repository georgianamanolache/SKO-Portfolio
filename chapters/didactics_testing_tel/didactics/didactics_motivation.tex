%Course material development for the \acrshort{ale} elective was fist and foremost stimulated by the course survey results shown in~\cref{appendices:survey}. 
%It was also intended to facilitate future course teachers to better understand the subject. 
%Furthermore, clear definition of the course learning outcomes was requiring immediate attention. 
%Along side the poor learning outcomes, the assessment criteria was essentially missing, thus, not fully adhering to the \acrshort{fhict} assessment standards.
%Lastly, alignment with the faculty formative assessment educational vision required further consideration. 
Among many professional skills an ICT graduate should develop and manage, soft skills have increased in demand within the IT sector recently~\cite{europeanFoundationalICTBOK2015}. These include, but are not limited to, communication, teamwork, decision-making, time management and verbal and written skills.
With the redesign of the \acrshort{fhict} curriculum architecture, the level of a unit of study is expressed in terms of the HBO-i framework~\cite{hboi} which links professional tasks and duties with professional skills~\cite{FHICTNewCurriculum}. However, there is little to no focus on the written skills.
\\\\
Regardless of the chosen profile, a \acrshort{fhict} undergraduate student faces a variety of writing tasks throughout the profile program. 
These tasks become progressively complex, from writing project documentation (e.i. a project plan or user requirements specification in early internal projects), to writing process reports that describe the applied research methodology (e.i. internship and graduation process reports with partners in education). 
Thus, it is important to guide students in exploring writing styles in order to become better writers.
\\\\
Therefore, writing skills module development is first and foremost needed to align with the entrance level and an exit level of the \acrshort{fhict} study units that have considerable writing tasks. While only Semester 3 has already been developed, this work also serves as proposal for new curriculum development for Semester 6, which is due to start in September 2021.


