In my opinion, research is an essential part of a modern \acrshort{ict} professional. 
While I knew about applied research as an alumna, I learned further about the integration of research throughout the entire four year study programme at \acrshort{fhict}. 
This has especially became obvious to me while teaching in the new curriculum in the academic year 2019-2020, since before it was less notable. 
As students advance in their study in the new curriculum, they learn progressively about applied research as a professional task. 
Thus, I believe that if consistently addressed, the research task will become easier for students in their internships and graduation internships, unlike old curriculum students, which were introduced to research later.\\\\
Furthermore, revising the \acrshort{dot} framework has made me a better supervisor. 
While I carry many aspects of reporting from the basic/fundamental research experience in my supervision, it is always good to revise the institute research framework. Lately, students have become more professional in employing the framework as a result of more research oriented courses in the new curriculum; however, I always find they lack soft skills. Especially for the first internship, students often struggle in documenting their process. I try to be as explicit and through in providing feedback, especially related to writing. Furthermore, I have learned to adapt and accept repetitive mistakes even after receiving thorough feedback. While I strive for all students to succeed in all aspects, I learned to become more accepting of students being less concerned of their writing skill and adapted my feedback accordingly. 
\\\\
Lastly, undergoing the task of portraying research expertise was one of my favourite tasks in this portfolio. The example presented shows just one of the situation where I employ my research skills in my teaching. As I plan to pursue a PhD career in the near future, personally, research is always an exciting and stimulating task.
