In any higher professional education context, the student population is expected to adopt an academic discourse to satisfy the requirements of successful communication. 
The discourse usually takes the form of presentations, feedback or writing. 
\\\\
Writing tasks constitute a major component in any higher professional education setting.
At \acrshort{fhict}, an undergraduate is expected to demonstrate writing skills as part of their professional skills defined by the HBO-i framework through a large variety of written tasks.  
These tasks become progressively complex, from writing project documentation (i.e. a project plan or user requirements specification in early internal projects), to writing process reports that describe the applied research methodology (i.e. process report/portfolio during internship and graduation with partners in education).
However, many students are either unaware of the importance of writing in academic discourse or do not know what exactly constitutes academic writing.
Studies such as \citeauthor{Lillis2001} and \citeauthor{Elander2006} have argued that students may fail to demonstrate satisfactory writing skills due to the university not explaining and teaching its discourse practices and conventions explicitly enough.
\\\\
This study aims to raise awareness amongst the \acrshort{fhict} teachers and offer suggestions as to better facilitate the students’ development of writing skills.
In order to establish a background and a frame of reference for the study, a review of a range of academic writing considerations is presented. 
After discussing the methodological approaches which underline the study, the findings are discussed.
Finally, the conclusion brings together the most relevant insights which emerged from the research and outline some implications for the current academic context. 