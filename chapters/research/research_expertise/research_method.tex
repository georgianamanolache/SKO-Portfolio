% \subsubsection{Data collection}\label{chapter:research_method_data_collection}

% An \acrshort{ale} student group that received both versions of the course with and without instructional materials was utilized.

%2020: 168 +  2019: 82 + 2018: 43 = 293
The investigation was carried out with \acrshort{fhict} teachers and undergraduate students. 
To identify teacher' experience with assessing writing tasks, 41 teachers with at least one year of experience within the \acrshort{fhict} education were given an Writing Skills Assessment  Questionnaire (appendix~\ref{appendix:writing_skills_assessment_questionnaire}).
To identify students’ perception and understanding of writing, three cohorts (2018, 2019, and 2020) with a total of 293 undergraduates were given a Writing Skills Questionnaire (appendix~~\ref{appendix:writing_skills_questionnaire}). 
These cohorts comprised students in their third-year and forth-year of studies enrolled in either internship, electives and graduation. 
In addition to close-response items to elicit information on students’ background and experience, there were several open-ended questions to acquire in-depth information on experiences of both assessors and students.
Some open-ended question asked directly ‘Can you state other writing consideration you know than those mentioned above? Try to state as many as possible.’ or 'Do you think students require further attention in improving writing skills? If yes, can you state what should be improved.'.
\\\\
Students' writing skills were investigated through the analyses of internship and graduation assessments. The analysis included 57 alumni (cohort 2017) were scores for assignment execution, writing, and presentation and defence were considered from both internship and graduation assignments. The analysis focused on identifying significant progress in students' writing based on prior experience from internship to graduation.
\\\\
As part of a wider investigation,
48 internship reports of third-year students (cohort 2019) were analysed. The analysis focused on the writing guidelines students referred in the development of their writing task. These guidelines include report writing guidelines handbook for internship and graduation  as well topics presented during the old curriculum course \acrfull{popd}~\cite{fhictpopd2} (applies to students who have started a study program at \acrshort{fhict} up to and including the spring of 2019).
