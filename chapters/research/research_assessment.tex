\subsection{Role as internship assessor}\label{subsection:role_internship_assessor}
At \acrshort{fhict}, the internship/graduation internship assessment is paired, summing the opinion of the supervisor and a second assessor evaluating only the final version of the internship report. Assessors align their evaluation on the intern's report and a general agreed assessment is formed. Then, the assessment is further discussed and aligned with the other parties, such as company supervisor and external assessor (only in graduation), at the time of the internship assessment (e.i. final presentation meeting).
\\\\
As a supervisor, a general assessment is usually formed from the first few feedback sessions. The intern I discuss in~\cref{chapter:research_supervisior} had a satisfactory report which with feedback proved great improvement in the final version. This is usually how the students perform in the writing part of the internship. I always pride myself with students that understand the importance of the written assessment and succeed to improve and deliver better documentation. After discussing the report with the second assessor, the writing was assessed to 8.5 out of 10, the student scoring overall a higher than average grade (considering work and presentation/defense), namely of 9 out of 10.
\\\\
Apart from assessing interns I supervised, I also took the role of second assessor in both internships and graduation internships.
While the second assessor role was introduced later in the internship assessment as it already existed in the latter, I believe that it is very favorable to assess reports other than the own supervised intern reports.
On the first semester of the academic year 2019-2020 I took this role for several students. My opinions on these reports is shown in~\cref{appendices:internship_report_appendix}. One student had a poor final process report, although the delivered work was sufficient. 
The internship assignment involved developing a prototype to fully automate a heavily manual operated application to then to test for viability against the original. 
After consulting the supervisor, the student was given a second chance to fix the report grade, scoring initially a 5 out of 10, which is a fail. 
First, the student delivered a journal like report, with weekly updates and implementation details which could have been better written in the appendix. 
Overall, the report read difficult and did not satisfy the \acrshort{fhict} standards. 
For instance, the research approach and research questions were not clearly stated. 
Also, the information about research and development was disrupted into two main sections instead of one process, proving that the student was unaware of how to apply the \acrshort{dot} framework. 
With the second attempt the student managed to incorporate some of the feedback and pass with a 5.5 out of 10. 
\\\\
Another student was assessed poorly due to the overall report structure, although the content was generally understandable.
The internship assignment concerned implementing different tasks to an already existing application, including bug fixes, new feature implementations, testing and experimenting with new tools and technologies.
While this time, this student had stated a research strategy, the research questions were missing. 
Furthermore, the report was structured as a journal and included implementation details which do not add the the overall understanding of the process. 
After consulting the supervisor, the student was assessed with 6 out of 10 for the report.
\\\\
At the end of the academic year 2019-2020 I was the second assessor of two graduation interns. My opinions on these reports is shown in~\cref{appendices:research_assessing_graduation}.
The first graduate intern was assessed in June 2020. 
The graduate developed a new format for survey specifications that reduces the complexity of code syntax, using spoken language instead. 
While the assignment was carried out during the COVID-19 outbreak, the graduate managed to achieve the main objectives in rather special circumstances (e.i. home work and little to none company interaction).
Since this was the first time I handled a graduation report, I initially observed my co-assessor's opinions about the writing.
I quickly learned that generally we shared the same opinions. 
While the writing was understandable, the main issue was that the graduate included details that should not be part of a process report, but better for a user manual. 
For example, the intern had snippets of code in a specific programming language, where usually algorithmic statements are presented in a process report in pseudo-code at most. 
Furthermore, the reading was complex and heavy in technical terms, which could have been eased with inclusion of more graphical representation of the processes. 
I was surprised to discover that the supervisor did not see a complete final draft of the report before. 
Given the immense improvement compared to the previous report version noted by to the graduate supervisor, the report was graded with 7 out of 10, which indicates generally an average performance (e.i. as expected).
\\\\
In August 2020 the second graduate intern was assessed. While the graduations take place before July in the second semester of the academic year, this intern had a delay due to health issues. The graduate conducted extensive research to develop a fully digitized medical whiteboard.
The report described the activities and the way of working sufficiently, however, more technical details would have been expected in the appendix. 
Furthermore, some issues, such as contractions, typos and grammar mistakes could have been easily avoided if the student would have double checked the text. After discussing with the supervisor, the report was assessed with 7 out of 10.
\subsection{Role as project assessor} \label{subsection:role_project_assessor}
The second semester students from ICT \& Software Engineering are introduced to research and the \acrshort{dot} framework during the execution of a group project. 
The project is a software management solution for a shop that keeps track of employees and products, on the basis of proper software design. 
The students research an optimal software architecture to accomplish the project requirements. 
Furthermore, the students work for a client under the supervision of a tutor, roles fulfilled by two different teachers. 
While the supervisor is observing the process and guiding groups through roadblocks when required, the client is not be aware of any of the technical aspects and meets the groups few times to specify project requirements.
I took the role of both client and supervisor for the academic year 2019-2020 second semester, which was the first iteration in the new curriculum. While I supervised three groups and acted as the client for other three groups, below I discuss groups from each role.
\\\\
As a project supervisor, I assess groups on their process, documentation and deliverables, verifying also the technical side of the delivered products. One of my groups had communication issues between the members, with two students performing outstanding and the other two slacking constantly. It is also worth mentioning that the students are sufferers of the COVID-19 outbreak online education, which has affected their group work routine. Initially the work and deliverables were poorer, however, based on feedback, the students improved. I focused especially on the architecture design (e.i. \acrshort{uml} diagram), being the core knowledge the students acquire in the ICT \& Software Engineering second semester. The group especially improved their soft skills, namely presentation and report writing, as I emphasised the importance of these skills for the \acrshort{fhict} assessment but also after the graduation. While the formative indications as a group progressed from S(Satisfactory) to O(Outstanding)\footnote{In the new curriculum letter grading is used instead: P(Poor), U(Unsatisfactory), S(Satisfactory), G(Good), O(Outstanding)}, the group even impressing the client with extra features, the slacking students were assessed lower, to just a S. An excerpt of this group's process report is shown in~\cref{appendices:research_assessing_project}.
\\\\
As a client, I assess the groups on how they conduct meetings, if they ask the right questions and if they deliver what they promised. 
Unlike the supervisor, I will assess products on a shallow level (e.i. only at the user experience). One of the groups that impressed me from the beginning with researched solutions and prototypes exceeded in my opinion the second semester learning goals. While I could see room for improvement in the documentation, this group was assessed O throughout all the three formative indication moments, satisfying all the requirements that were asked from me as a client. In contrast, another group underdelivered throughout the project, despite the continuous feedback and support. Initially formed out of four students, the group dissolved during the project execution, leaving only one member active. This was a challenging group to assess, however, together with the project tutor it was decided to lower the project requirements and give the student a chance to achieve at least a S, and, thus, pass the project. While the majority of the work was already done by the one student which was left in the group, the student was suggested to focus on testing already existing features and deliver a robust solution with less functionality than initially promised. 
\\\\
It is challenging for me to act as a client in these projects as I am always driven to tutor students when they can improve. For the poor group for which I acted as a client, it was difficult for me not give tutor feedback, however, at each formative indication moment I discussed improvements with the tutor and shared our feedback collectively. Examples from my assessment feedback for these groups is shown in~\cref{appendices:research_assessing_project}.