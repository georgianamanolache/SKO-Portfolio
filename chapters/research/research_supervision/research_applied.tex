Unlike academic research, applied research is of practical nature and is about applying existing knowledge. 
Such research starts with a question and ends with an answer, a solution.
At \acrshort{fhict} examples of research questions are:
\textit{What is the most suitable programming language to use?} or 
\textit{Which data should be used in the database system?}.
Students perform applied research to investigate options and conclude which option is the best fit.\\\\
The applied research model has a basic flow.
First, a problem definition is used as input (e.g. a question that must be answered to solve the problem).
Then, different tasks to research a possible solution are performed, deciding what is the most appropriate and start realizing the solution. 
This phase is also called innovation-space as here the actual work (innovation) takes place (e.i. results from the research are applied here to achieve the desired solution).
Finally, the desired solution is achieved as output to the research. 
For instance, deliverables, a typical output, function as proof that the desired solution is achieved.
These steps all together define applied research.
To aid students in the research process, the \acrfull{dot} that gives a framework to perform applied research is selected, its structure being described in the following section.




