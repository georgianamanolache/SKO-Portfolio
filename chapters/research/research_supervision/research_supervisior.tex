Yearly, I supervise internship students in various assignments, ranging from a more research nature to plain engineering. At \acrshort{fhict} internships are mainly external, where students can gain further experience by performing in a real work setting.
As internships are external, communication with the intern is essential to bridge client requirements and provide appropriate guidance.
Thus, the role as a internship supervisor is essentially to guide the interns in the process of understanding requirements, employing research strategies and reporting the process. 
\\\\
In this section, the supervision of an internship student previously unknown to me from the 2020 Fall semester is presented. 
I quickly derived from the first meeting that the student was proficient technically but lacked soft skills; however, was highly motivated. 
As the internship assignment was research heavy, the student expressed his concern about the planning. 
The assignment implied mainly researching  and creating a 3D head reconstruction  software, that uses pictures  of a  user  to output a 3D model for the \acrshort{vr} environment avatars. 
This feature would provide a realistic interacting between \acrshort{vr} users, preferred in the \acrshort{vr} entertainment and training endorsed by businesses.\\\\  
It is worth mentioning that throughout the internship I had a very good communication with the student via \Gls{mteams}, with report updates every two weeks.

\subsubsection{Understanding requirements}

Following the first meeting with the student where the  internship assignment was generally introduced, the first draft of the project plan was the next piece of information which helped me understand the assignment further. 
A project plan is a kind of contract between the client and the student. 
While the student needs to work autonomously during the internship project, the company guides the student in this process and needs to approve the project plan. 
The intern also discusses the project plan with the university supervisor.
This document is typically written in the first few weeks of the internship period, where the intern learns more about the company and about the assignment. 
Apart from the assignment context, the project plan includes phasing, risk analysis and a communication plan.
As an internship supervisor, I am mainly involved with the structure of the project plan and the way in which the assignment is phased. 
Also, whether the assignment objectives will be achieved is especially one of the aspects a university supervisor is concerned with.\\\\
The intern doubted whether all the assignment objectives will be achieved. 
Initially, the assignment comprised two main deliverables: a server-hosted application which can use photos of person to create and store a 3D reconstructed head model and a a mobile application that can accommodate the photo management for the server-hosted application. 
However, the project goal was mainly to create a realistic player recognition method that reconstructs the head of a person from a picture in 3D. 
Furthermore, reconstructing a 3D model from a picture required extensive research, making it already a complex task alone.
Also, the assignment context was vaguely described. While the project plan was still a fist draft, the intern had to further clarify the assignment context with the company.\\\\
As a supervisor, I had to ensure that the intern can construct a project plan accordingly. 
My main concern was the deliverables workload which could have easily resulted into two potential stand-alone internship assignments. 
As the task of 3D model reconstructing from a 2D image had a true research nature, I suggested to the intern to prioritize this in the assignment instead, since it matched the project goal. 
While the intern seemed highly motivated, a narrow assignment would better indicate realistic achievements of the internship. 
Furthermore, the intern already had optional deliverables planned for when the main task was accomplished. 
The intern confirmed shortly, after discussing assignment objectives with the company, that the focus is indeed on 3D model reconstruction, as there are no commercially available solution to perform this task and further research is required, potentially leaving less time for the other initially planned tasks.\\\\
Another concern was the formulation of the assignment context, namely current situation and assignment  problem. 
While I have an overall understanding of the technologies the intern was presenting in these segments, the project plan as well as the process report are deliverable that should be comprehensive to inexperts. 
Thus, detailed explanation of specific technologies is compulsory for a \acrshort{fhict} internship deliverable.\\\\
Sections of the the initial and final version of the project plan are shown in~\cref{appendices:research_supervising}. 
The initial version improved significantly and narrowed the intern task. This resulted into a new phasing which the intern took care accordingly. 

\subsubsection{Employing research strategies}

Regardless of the internship task, research is inevitable, and unlike academia, within \acrshort{fhict} the research is applied. 
Given the nature of the intern's assignment, the initial phase of the internship was shifted towards the research. 
The first version of the process report gave me the first insight into the intern's research approach. 
The progress report is the document interns write for the university and are assessed upon.
As the name suggests, the focus of this report will be on the process, namely how interns have come to a certain result, therefore, demonstrating that they are accountable for their actions during the assignment, and that they are able to underpin their decisions. 
Furthermore, the writing should be according to the university standard.
As a supervisor, it is my responsibility to ensure intern students have a good understanding of the research methods and their purpose. \\\\
While the intern has followed the university standards for the most part, including the \acrshort{dot} framework for applied research, the research approach was not clearly stated. 
Firstly, the research questions were missing the student presenting directly the solution. 
Each subsection of the research phase documentation could clearly be comprised into a research question, which made me comprehend that the intern had a sense of the research task but did not state the question. 
Research questions are essential to the research process. 
By defining exactly what the researcher is trying to find out,  these questions do not only dictate the trajectory of the rest of the steps taken to conduct the research but they also help anyone who may be interested in the topic to better grasp the research content. 
While they do not necessarily define where the research will end up they provide focus where the research first started.\\\\
Furthermore, the application of the \acrshort{dot} framework was incorrect, the intern stating his approach at the end of each documentation of the research task (e.i. field and lab), after specific research methods were already mentioned (e.i. interview and testing).
Stating the research approach follows right after the research questions. 
Research approaches are the plans and the procedures for research, strategies and specific methods. 
More importantly, this section is intended to show that the researcher has thought carefully about the link between the research questions and selected research methods.\\\\
In my opinion, how well a student has formulated the research approach section illustrates that the student has carefully designed and produced a sound research. 
Thus, my task was to guide the intern in formulating the main research question and sub-question and state the corresponding approach. 
I instructed the student how to formulate the research approach section, with excerpts of the feedback shown in~\cref{appendices:research_supervising}.
\\\\
It is always challenging for the interns to properly apply the \acrshort{dot} framework and document the approach. 
I have struggled myself with the framework as an \acrshort{fhict} alumna since before there was little to no instruction before the execution of the internship. 
Recently, with the introduction of personal orientation and personal development courses as well as professional skills courses in the first semesters \cite{FHICTNewCurriculum}), the knowledge of the \acrshort{dot} framework has become available much earlier. However, as a teacher of these courses, I realized students even then do not devote much effort to soft skills, being mainly interested in the technical courses.
Furthermore, there is a research based course before the graduation semester, which focuses on the application of the framework. However, this framework is also required in the internship which comes much earlier in the study. Ever since I started working within \acrshort{fhict} I highlighted the importance of applied research to all semester students and I hope that with the new curriculum, students will better carry this aspect of the education throughout their study. 
\subsubsection{Reporting the process}
At \acrshort{fhict}, during internship semesters, the students are required to write a portfolio or process report. 
In this case, the intern chose to document his process in a report format as this type of document felt more comfortable.
Essentially, a report is a short, sharp, concise document which is written for a particular purpose and audience.
Any report - whether it is about a or describing the processes of conducting research in a company - is meant for a particular type of audience. 
The audience for the internship process report is mainly \acrshort{fhict}. 
Thus, the writing has to comply to the structure and standards required by the institute. 
This involves the writing style, what to include, the language to use and the length of the document. 
Furthermore, the purpose of this report is to demonstrate the process, from understanding the problem and research questions to achieving desired results and reflecting on them.
\\\\
The intern sent regular progress updates of his document, easing the task of tracking of the process. 
Given the nature of the interns assignment, the documented progress would imply new short-term outcomes which would determine the next steps in the process.\\\\
My role as a supervisor was to provide formative feedback on each report version. More specifically I guided the intern on the following aspects of the report writing: audience, purpose, strategy, organization, style and flow. Excerpts of of the feedback are shown in~\cref{appendices:research_supervising}.
\\\\
Audience, purpose, and strategy are typically interconnected. 
When the audience knows less than the writer, the writer’s purpose is often instructional/informative. 
When the audience knows more than the writer, the writer’s purpose is usually to display familiarity, expertise, and intelligence. 
Commonly, an intern student writes a combination of both situations. 
The students need to show expertise in applying research, but may also instruct expert knowledge related to the work to help the reader fully grasp the content.
Initially, the intern composed an introductory paragraph that was well structured, however, required expert knowledge. While I am myself familiar with some of the utilized technical terms, the overall audience of a \acrshort{fhict} process report is an \acrshort{ict} professional that is fairly informed about existing technology but not necessarily expert. Therefore, specific terms and technologies require further explanation to fully understand the content.
\\\\
Readers have the expectation that information will be presented in a structured
format that is appropriate for the particular type of text. At \acrshort{fhict} the internship report should be organized in specific chapter and sections, such as \textit{Introduction}, \textit{About company} and \textit{Process}. While the intern presented the progress results in sections of the \textit{Process} chapter, the main outcome of the internship was difficult to locate. Therefore, a final \textit{Result} chapter that summarizes the main outcomes is necessary, especially to prove whether the research question and sub-questions were answered. The intern managed to formulate this chapter after several feedback meetings.
\\\\
Furthermore, students need to be sure that their communications are written in the appropriate style. 
A formal report written in informal English may be considered simplistic, even if the actual ideas and/or data are complex.
Moreover, students have to be aware of the stylistic advice within the institute.
The intern had an expected stylistic communication in the report.
For example, some processes and procedures were not described with passive voice. 
Although aware, sometimes the intern used contractions (short forms) \textit{don't}, \textit{aren't}, \textit{shouldn't}.
In the case of the \textit{I} pronoun, the intern appeared to overuse it, while this is usually present in an \acrshort{ict} related report to support and emphasise own point or outcome. I highlighted these stylistic suggestions once and let the intern decide whether or not they will be modified/used in the rest of the writing.
\\\\
Another important consideration for successful communication is flow,
moving from one statement to another. Naturally, establishing a
clear connection of ideas is important to help the reader follow the text. Linking words, such as \textit{furthermore}, \textit{moreover}, \textit{however} usually provide this flow. The intern lacked in linking words and had some misuse of punctuation, which were specified in the feedback.
\\\\
With every internship I take an instructional position in the report writing process. I especially take this instructional position during the first internship to help students better prepare for the graduation internship report. This intern was always eager to receive and incorporate feedback. While the intern was already at a satisfactory level, I provided feedback on every writing aspect that could be improved and let the intern decide whether to use my recommendations or not. I am content with the intern's progress in terms of writing which will be essential not only in the following internship, but also after graduation.






