Education is the foundation of our society's knowledge~\cite{FKO}.
Education is provided primarily through educational institutions (from preschools and elementary schools to universities) by a pedagogue. 
The role of a pedagogue is to convey knowledge to scholars in an appropriate, motivating manner. 
Furthermore, a pedagogue can further inspire colleagues through own didactical experience and practice.
\\\\
At Fontys, the employees performing the role of a pedagogue can demonstrate didactical experience by obtaining the \acrfull{fko}. 
Every new teacher employee without a teaching qualification is required to obtain at least the \acrfull{bko}~\cite{FKO}.
The following mandatory qualification which professionalizes didactics is \acrfull{mko}. The final qualification, \acrfull{sko}, shows a teacher's experience in proposing and developing new valuable educational modules whiting the current didactic and educational context, while instructing and inspiring other colleagues.
\\\\
In this portfolio, personal experience on the basis of didactics, testing, \acrfull{tel} and research activities within \acrfull{fhict} is presented as proof for the \acrshort{sko}.
%First, a profile sketch is presented in~\cref{chapter:profile}, where background information followed by own roles and acquired experience throughout the activity at \acrshort{fhict} is described.
\\\\
Didactics are presented in the following three chapters.
\Cref{chapter:didactics} describes the  development of new educational module development  and  self-reflection
the \acrshort{fhict} education.
\Cref{chapter:testing} describes the assessment choices and construction of tests and test materials for the developed module, followed by reflection on own findings and opinion on what can be improved further. 
\Cref{chapter:tel} describes own proficiency of utilizing technology in education, concluded with personal opinion over the recent technological advancements.
\\\\
\Cref{chapter:research} presents the research activities supported by concrete examples in supervising and assessing research at \acrshort{fhict}. 
%Furthermore, own research expertise is displayed through a study which tries to identify student performance trends on the basis of proper course materials.
%\\\\
%Finally, \cref{chapter:conclusions} concludes the personal professionalization within didactics and research, resulted from development to improve the quality of education at \acrshort{fhict}.
%To support the work presented throughout each chapter, the appendices enclose examples of various documents and activities. %relevant to the portfolio.
%These are referred in the corresponding chapters.
